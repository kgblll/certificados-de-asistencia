\documentclass[12pt]{article}
\usepackage[utf8]{inputenc}
\usepackage[spanish]{babel}
\usepackage{mathpazo}
\renewcommand{\familydefault}{\rmdefault}
\usepackage[landscape,a4paper]{geometry}
\geometry{verbose,tmargin=0cm,bmargin=0cm,lmargin=0cm,rmargin=0cm}
\usepackage{fancybox}
\usepackage{calc}
\usepackage{multicol}
\usepackage{graphicx}
\usepackage{url}
\usepackage{eso-pic}

\newcommand\BackgroundPic{%
\put(0,0){%
\parbox[b][\paperheight]{\paperwidth}{%
\vfill
\centering
\includegraphics[width=\paperwidth,height=\paperheight,%
keepaspectratio]{background1.jpg}%
\vfill
}}}


\begin{document}
\AddToShipoutPicture{\BackgroundPic}
~
\vspace{1.2cm}
~
\begin{center}

\begin{table}[h]
\footnotesize
\begin{center}
\begin{tabular}{lr}
~\hspace{0.7cm}
\includegraphics[height=2.4cm]{urjc.png} ~\hspace{3cm}~&
\includegraphics[height=1.8cm]{fecyt.png}
\end{tabular}
\end{center}
\end{table}

\LARGE{El Departamento de Sistemas Telemáticos y Computación (GSyC) de la\\
Escuela Técnica Superior de Ingeniería de Telecomunicación} \\
\Large{tiene el honor de otorgar el presente}\\
\vspace{1cm}
\fontsize{50}{60}{\textbf{CERTIFICADO}}}

\vspace{0.4cm}

\Huge{a \textbf{%pointname
}}

\vspace{0.4cm}

\Huge{por su asistencia al
% \textbf{%pointdni}
}

\vspace{0.3cm}

\Huge{\textbf{TALLER DE DR. SCRATCH PARA DOCENTES}}

\vspace{0.3cm}

\Large{celebrado el viernes, 27 de febrero de 2015 de 18 a 20 horas en Medialab-Prado}

\vspace{.2cm}

%\Large{Y para que as\'i conste se expide el presente certificado}
%
\vspace{1cm}

\begin{table}[h]
\footnotesize
\begin{center}
\begin{tabular}{lp{2cm}cp{2cm}r}
%\includegraphics[height=3cm]{marca_UOC.png} ~\hspace{3cm}~
%&
%\includegraphics[height=3cm]{Logo_urjc.PNG} 
%& 
\Large{ } & & 
\Large{  } & & 
\Large{Dr. Gregorio Robles Martínez} \\
Sello del Departamento & &
 & & 
Director del proyecto \\ 
\end{tabular}
\end{center}
\end{table}

\vspace{-0.5cm}

\scriptsize 
{La herramienta Dr. Scratch tiene como objetivo ofrecer una vía de aprendizaje y realimentación sobre la calidad de los proyectos desarrollados en Scratch. \\ 
El taller ha sido organizado con el apoyo y la colaboración de la Fundación Española para la Ciencia y la Tecnología (FECyT). \\ 
El taller ha sido posible también debido a la colaboración de Programamos.es, Medialab-Prado, y se encuentra dentro de las actividades del Google RISE Award 2015 recibido por el equipo organizador de la URJC.}
%Diploma elaborado con el software libre \texttt{Pyploma}, (c)
%\url{fjruizruano@gmail.com} bajo licencia
%GPLv3. Visite: \url{http://code.google.com/p/pyploma}

\end{center}
\end{document}
