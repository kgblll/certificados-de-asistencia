\documentclass[12pt]{article}
\usepackage[utf8]{inputenc}
\usepackage[spanish]{babel}
\usepackage{mathpazo}
\renewcommand{\familydefault}{\rmdefault}
\usepackage[landscape,a4paper]{geometry}
\geometry{verbose,tmargin=0cm,bmargin=0cm,lmargin=0cm,rmargin=0cm}
\usepackage{fancybox}
\usepackage{calc}
\usepackage{multicol}
\usepackage{graphicx}
\usepackage{url}
\usepackage{eso-pic}

\newcommand\BackgroundPic{%
\put(0,0){%
\parbox[b][\paperheight]{\paperwidth}{%
\vfill
\centering
\includegraphics[width=\paperwidth,height=\paperheight,%
keepaspectratio]{background1.jpg}%
\vfill
}}}


\begin{document}
\AddToShipoutPicture{\BackgroundPic}
~
\vspace{1.2cm}
~
\begin{center}

\begin{table}[h]
\footnotesize
\begin{center}
\begin{tabular}{lr}
~\hspace{0.7cm}
\includegraphics[height=3cm]{urjc.png} ~\hspace{8.2cm}~&
\includegraphics[height=3cm]{gsyc.png}
\end{tabular}
\end{center}
\end{table}

\LARGE{El Departamento de Sistemas Telemáticos y Computación (GSyC) de la\\
Escuela Técnica Superior de Ingeniería de Telecomunicación} \\
\Large{tiene el honor de otorgar el presente}\\
\vspace{1cm}
\fontsize{50}{60}{\textbf{DIPLOMA}}}

\vspace{0.4cm}

\Huge{a \textbf{%pointname
}}

\vspace{0.4cm}

\Huge{por su \textbf{%pointdni
}
en el }

\vspace{0.3cm}

\Huge{\textbf{CAMPEONATO VIRTUAL DE SHELL Y PYTHON}}

\vspace{0.3cm}

\Large{celebrado del 26 de septiembre al 5 de octubre de 2013}

\vspace{.2cm}

\Large{Y para que as\'i conste se expide el presente diploma a fecha de
\today.}

\vspace{1cm}

\begin{table}[h]
\footnotesize
\begin{center}
\begin{tabular}{lp{1cm}cp{1cm}r}
%\includegraphics[height=3cm]{marca_UOC.png} ~\hspace{3cm}~
%&
%\includegraphics[height=3cm]{Logo_urjc.PNG} 
%& 
\Large{Dr. Jesús M. González Barahona} & & 
\Large{Dr. José Centeno González} & & 
\Large{Dr. Gregorio Robles Martínez} \\
Profesor de la asignatura & &
Director de Departamento & & 
Profesor de la asignatura \\ 
\end{tabular}
\end{center}
\end{table}

%\vspace{1cm}

%\scriptsize 
%El Campeonato Virtual de Shell y Python tuvo un total de 21 participantes de entre los alumnos de la asignatura "Protocolos para la Transmisión de Audio y Vídeo por Internet". \\
%FLEQ (Free Libresoft Educational Quizbowl) es software libre para
%organizar torneos virtuales, desarrollado por la Universidad Rey Juan Carlos.
%Visite: \url{http://trivial.libresoft.es}\\
%Diploma elaborado con el software libre \texttt{Pyploma}, (c)
%\url{fjruizruano@gmail.com} bajo licencia
%GPLv3. Visite: \url{http://code.google.com/p/pyploma}

\end{center}
\end{document}
